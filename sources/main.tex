\documentclass{article}
\usepackage[utf8]{inputenc}

\usepackage[T2A]{fontenc}
\usepackage[utf8]{inputenc}
\usepackage[russian]{babel}

\usepackage{multienum}
\usepackage{geometry}
\usepackage{hyperref}

\geometry{
    left=1cm,right=1cm,
    top=2cm,bottom=2cm
}

\title{История}
\author{Лисид Лаконский}
\date{February 2023}

\newtheorem{definition}{Определение}

\begin{document}
\raggedright

\maketitle
\tableofcontents
\pagebreak

\section{Практическое занятие №5, «становление Российской империи»}

\subsection{Эволюция московского царства в Российскую империю (вторая половина 17 — первая четверть 18 века)}

В \textbf{1697 году} Швецию возглавил пятнадцатилетний \textbf{Карл XII}.

У многих ее соседей имелись к ней различные претензии: \textbf{датско-норвежское королевство} давно соперничало с Швецией за господство на Балтийском море, влияние на Шлезвиг-Гольштейнское княжество.

\textbf{Курфюрст Саксонии и король польский Август II} хотел упрочить свою власть в Речи Посполитой, вернув \textbf{Ливонию}, отошедшую Швеции в результате подписания \textbf{Оливского мирного договора} в \textbf{1660 году}

\textbf{Московское царство} также было заинтересовано в ослаблении шведских позиций в Балтийском регионе.

\hfill

Все это в итоге привело к заключению \textbf{21 ноября 1699 года} \textbf{Преображенского договора} — союзного договора между русским царем Петром I и саксонским курфюрстом и королем Речи Посполитой Августом II Сильным.

И заключив \textbf{14 июля 1700 года} \textbf{Константинопольский мирный договор} с Турцией (что стало итогом \textbf{Азовских походов}), Петр I на следующий же день после того, как узнал о том, что договор был заключен, \textbf{30 августа}, объявил Швеции войну.

\hfill

Начало войны задалось плохо: осада Риги саксонским корпусом прошла неудачно, Дания же была вынуждена \textbf{18 августа} заключить мирный договор с Швецией и отказаться от союза с Августом II.

Говорить о Северной войне можно много: все-таки, длилась она двадцать один год, но в контексте темы неуместно говорить о всех действиях, что тогда происходили. Наиболее важны итоги.

\hfill

\textbf{Ништадский мирный договор} между Русским царством и Шведским королевством, завершивший Северную войну, был подписан \textbf{10 сентября 1721} года в городе Ништадте.

Война закончилось победой антишведской коалиции, и победа в Северной войне выдвинула Россию в число крупнейших европейских государств.

По итогам войны к России были присоединены Ингрия, Карелия, Эстляндия, Лифляндия, основан Санкт-Петербург, Российское влияние утвердилось в Курляндии.

Была решена задача, поставленная Петром I — обеспечение выхода к морю и налаживание морской торговли с Европой

\hfill

И в \textbf{сентябре 1721 года} Сенат и Синод решили преподнести Петру титул Императора Всероссийского, подобно тому, как такие титулы в дар приносились Римским сенатом правителям за знатные дела.

А \textbf{2 ноября 1721 года} в церкви Святой Троицы состоялась церемония, на которой канцлер Сената произнес торжественную речь о провозглашении титула «Отца Отечества, Петра Великого, Императора Всероссийского»

\pagebreak
\subsection{Утверждение абсолютизма в России в начале 18 века}

\begin{definition}
    Абсолютная монархия — разновидность монархической формы правления, близкой к диктатуре, при которой вся полнота государственной (законодательной, исполнительной, судебной, военной), а иногда и духовной (религиозной) власти находится в руках монарха.
\end{definition}

Абсолютизм характеризуется тем, что прекращает деятельность органов, существующих в сословно-представительной монархии (Земский собор, Боярская дума), и государственная власть получает большую самостоятельность по отношению к обществу, что и делает Петр, заменив Боярскую думу командой единомышленников. В 1699 г. была создана \textbf{Ближняя канцелярия} (административно-финансовый контроль в государстве). Заседания все более сокращавшейся Боярской Думы стали происходить в Ближней канцелярии. В 1708 г. в заседаниях Думы участвовало обычно 8 человек, которые управляли различными приказами. Это совещание получило название \textbf{Консилии министров}, фактически это был Верховный орган власти, который в отсутствие царя управлял не только Москвой, но и всем государством. Бояре и судьи оставшихся приказов должны были съезжаться в Ближнюю канцелярию по три раза в неделю для решения дел.

После учреждения \textbf{5 марта 1711 года} Правительствующего сената как высшего органа государственной власти и законодательства Консилия министров и Ближняя канцелярия прекратили свое существование

\hfill

Далее Петр изменяет \textbf{порядок престолонаследия}. Указом 1722 года он утверждает за собой право назначения преемника, отныне престолонаследие не связано с родством

В 1717-1722 гг. на смену 44 приказам конца XVII в. пришли \textbf{коллегии}. В отличие от приказов, система коллегий предусматривала разделение администрации на определенные ведомства, это создавало \textbf{более высокий уровень централизации}.

\hfill

Императору Петру I принадлежала \textbf{законодательная и исполнительная власть} в государстве. Он был последней и высшей инстанцией в решении судебных дел. Он являлся верховным главнокомандующим войсками и фактически главой русской церкви: в 1721 г. была образована Духовная коллегия, преобразованная затем в 1722 г. в Святейший правительствующий синод, который уравнивался в правах с Сенатом и \textbf{подчинялся непосредственно царю}.


\pagebreak
\subsection{Эволюция внешней политики России: от решения национальных задач к задачам имперским}

Много всякого хорошего написано в документике по ссылке \url{https://www.calameo.com/read/0067242684eca60d187ee}


\pagebreak
\subsection{Юридическое оформление Империи в первой четверти столетия}

\subsubsection{Первый источник, сомнительного качества}

Начало XVIII в. в России ознаменовалось \textbf{переходом к абсолютной монархии} и связано с деятельностью Петра I (1672—1725), который начал править формально вместе с братом Иваном, вскоре умершим. Данный период характеризуется крупными изменениями в российской государственности и всей системе общественных отношений.

Петр \textbf{принципиально отказался от византийской «двоицы», т.е. от двуединства власти императора и патриарха, сосредоточив всю власть в руках монарха и упразднив институт патриаршества}. Это не означало умаления роли Русской православной церкви в жизни российского общества. Это только расставляло акценты в отношениях между монархом и руководством РПЦ, касающихся вопросов государственной власти.

\hfill

Говоря об абсолютизме, надо иметь в виду, что понятия «самодержавие» и «абсолютизм» не идентичны. Абсолютизм представляет собой определенную стадию монархического правления, характеризующуюся наивысшей степенью централизации власти, ее бюрократизации, подчинением воли монарха, либо просто сведением на нет деятельности представительных органов, усилением полицейского контроля.

\hfill

Российский абсолютизм, сложившийся позже западноевропейского, имел ряд принципиальных отличий. Европейский абсолютизм явился следствием развития буржуазных отношений, на фоне научно-технического прогресса и освобождения крестьян от феодальной зависимости. \textbf{Российский абсолютизм, напротив, основывался на усилении крепостной зависимости в условиях научно-технического отставания}.

\hfill

Если Западная Европа делала ставку на развитие частной инициативы, то России для преодоления отставания, вызванного многолетней борьбой с монгольским нашествием, приходилось всячески усиливать административно-командную систему. Если европейские монархи находили опору в городах, то \textbf{российское самодержавие опиралось главным образом на крепостническое дворянство}. Европейский абсолютизм не был чужд либерализации, в то время как в России она подавлялась самым жестким образом. 

\hfill

Петровские реформы сильно всколыхнули Россию: \textbf{стало развиваться мануфактурное производство, активизировались крестьянская торговля и отхожие промыслы, получили развитие металлургия и кораблестроение}. Однако в России не было свободного движения рабочей силы. Промышленное производство в качестве рабочей силы обеспечивалось за счет \textbf{посессионных крестьян}, т.е. приписанных к посессионным мануфактурам, купленных к фабрикам, «вечноотданных» по указу от 7 января 1736 г. Этих крестьян нельзя было продавать отдельно от мануфактуры.

\hfill

После смерти патриарха Адриана (1700) новый патриарх избран не был, царским указом рязанский митрополит Стефан Яворский был назначен «экзархом святейшего Патриаршего престола, блюстителем и администратором». В 1721 г. царским Манифестом \textbf{патриаршество было официально отменено, а церковь подчинена Святейшему Правительствующему Синоду} (Духовной коллегии) во главе с обер-прокурором. Фактически главой церкви становился император, который решал все ее вопросы, кроме касающихся церковной доктрины и веры. Данная церковная реформа не пошла на пользу церкви, поскольку фактически включила ее в систему спецслужб, обязав священников информировать власти о том, что говорилось верующим на исповеди. Это ставило священнослужителя перед дилеммой: либо совершить тяжкий грех перед Богом, либо навлечь государственную кару.

\pagebreak
\subsection{Внешняя политика России в 1725–1762 годах: направления, успехи и неудачи}

\subsubsection{Первый источник, какой-то сайт школьного уровня}

\textbf{Отсутствие войн}. Незначительные боевые действия вел в Закавказье корпус князя Долгорукова.

В европейских делах был взят курс на установление союзных отношений с Австрией. С ней в 1726 году был заключен \textbf{Венский договор}. Целью сближения были совместные боевые действия против Османской империи. Она представляла собой еще достаточно сильного противника.

Поддержка на севере Германии герцога Голштейна в его противостоянии с Данией. С этим герцогом вступила в брак Анна, дочь Петра I. Это было новым явлением во внешней политике, так как в начале XVIII века расстановка сил в балтийском регионе была иной. Россия находилась в союзе с Данией против Швеции и Голштейна.

\subsubsection{Второй источник, википедия, непонятные источники}

За 2 года правления Екатерины I Россия не вела больших войн, только на Кавказе действовал отдельный корпус под началом князя Долгорукова, стараясь отбить персидские территории, пока Персия находилась в состоянии смуты, а Турция неудачно воевала с персидскими мятежниками. В Европе Россия проявляла \textbf{дипломатическую активность в отстаивании интересов голштинского герцога} (мужа Анны Петровны, дочери Екатерины I) против Дании. Подготовка Россией экспедиции для возврата герцогу Голштинскому отнятого датчанами Шлезвига привела к военной демонстрации на Балтике со стороны Дании и Англии. 

\hfill

Другим направлением русской политики при Екатерине было \textbf{обеспечение гарантий Ништадтского мира} и \textbf{создание антитурецкого блока}. В 1726 году правительство Екатерины I заключило Венский союзный договор с правительством Карла VI, ставший основой русско-австрийского военно-политического альянса второй четверти XVIII века.

\pagebreak
\subsection{«Великий прорыв» Екатерины II на южном и западном направлениях}

\subsubsection{Первый источник, про крым, причерноморье и северный кавказ}

В правлении императрицы Екатерины II российская внешняя политика приобрела исключительную активность, связанную, прежде всего с \textbf{присоединением причерноморских территорий}. Это направление внешнеполитического курса привело к двум русско-турецким войнам: в 1768-1774 и 1787-1791 гг. Россия решала назревшую национальную задачу воссоединения земель, входивших в состав Древнерусского государства. Ею также была оказана существенная поддержка славянскому освободительному движению Балканского полуострова, завоеванного турками в конце XIV века.

В \textbf{первой русско-турецкой войне} русская армия добилась ряда важных побед: А.В. Суворов провел у Туртукая в устье Дуная в мае-июне 1773 года успешные боевые операции, а 9 июня 1774 года во встречном 10 часовом сражении при Козлудже сломил сопротивление турок, а заодно и волю турецкого правительства к продолжению войны.

И в 1774 году Россия подписала выгодный для нее \textbf{Кючук-Кайнарджийский мир с Турцией}. России предоставлялось право торгового мореплавания по Черному морю и через проливы Босфор и Дарданеллы. На северном Кавказе к России отходила Кабарда, а Крымское ханство объявлялось независимым от Турции, при этом Керчь, Еникале и Кинбурн передавались в русское владение.

\hfill

В 70-80 годы XVIII века Россия наращивала свои внешнеполитические успехи. В 1779 году она выступила на \textbf{Тешенском конгрессе} в роли гаранта равновесия в Германии, посредничая в австро-прусском соперничестве. В 1780 году объявила \textbf{«Декларацию вооруженного нейтралитета»}, защищая право свободного мореплавания, стесненного в этот период Англией, воевавшей в Северной Америке со своими восставшими колониями. Инициативу России поддержали почти все страны Европы и только что возникшие Соединенные Штаты Америки.

В 1781 году подписан \textbf{союзный договор с Австрией}. В дипломатии и внешней политике, однако, по-прежнему доминировала задача реализации одной из важнейших национальных целей - присоединить Крым и причерноморские земли. Турция не хотела примериться с потерей Крыма и готовилась к новой войне с Россией. 8 апреля 1783 года Екатерина II подписала \textbf{манифест о присоединении Крыма к Российской империи}. Вытеснение Турции из Крыма укрепляло обороноспособность России на южных границах и способствовало ускорению экономического развития южнорусских и украинских территорий.

\hfill

Укрепились позиции России и на Кавказе. 24 июля 1783 года был подписан \textbf{договор царя Восточной Грузии (Карталии и Кахетии) Ираклия II с Россией о вступлении под российское покровительство} (протекторат). В 1787 году вассалом России признал себя владетель \textbf{Северного Дагестана} шамхал Тарковский, русского покровительства добивались царь \textbf{Западной Грузии} (Имертии) Соломон I и крупнейший \textbf{азербайджанский} правитель Фатали-хан Кубинский.

\hfill

В 1787 году началась \textbf{новая русско-турецкая война} (1787-1791). Поводом к ней послужили споры из-за статуса дунайских княжеств Молдавии и Валахии, где проживало христианское население, в пользу которого Россия имела право делать представления по договору 1774 года. Турция к тому же не хотела признавать перехода Крыма к России и русский протекторат над Грузией.

В 1788 году \textbf{начала военные действия против России и Швеция}, добивавшаяся реванша - за поражение, а Северной войне. Хотя в союзе с Россией была малоинициативная в военном деле Австрия, ее положение было трудным. Но соединенные русские и австрийские войска, руководимые А.В. Суворовым, разгромили турецкую армию. Успешно действовал на Черном море и недавно созданный русский военный флот адмирала Ф.Ф. Ушакова. На северном театре войны - против Швеции - также обозначился перевес русских сил. Шведы запросили мира, который и был подписан 3 августа 1790 года. \textbf{Верельский мирный договор} сохранял за сторонами их довоенные границы. А война с Турцией между тем продолжалась.

Союзник России Австрия в сентябре 1790 года заключила перемирие с турками, что резко ограничивало полосу для наступательных действий русской армии лишь Нижним Дунаем, где у турок имелась мощная крепость Измаил. Русское командование приняло решение штурмовать этот неприступный оплат Турции на Дунае. Штурм начался утром 11 декабря 1790 года и \textbf{завершился полной победой к середине того же дня}. Штурм Измаила был подвиг русских солдат и офицеров, плодом полководческого искусства А.В. Суворова. 

Война закончилась подписанием 29 декабря 1791 года \textbf{Ясского мирного договора}, подтверждавшего Кючуг-Кайнарджийский трактат 1774 года и все последовавшие за ним дипломатические акты. Ясский мир обеспечил России широкий доступ к Черному морю и экономические связи со странами среднеземноморского бассейна становились все более тесными.

\subsubsection{Второй источник, про разделы речи посполитой}

В состав федеративного польско-литовского государства \textbf{Речь Посполитая} входили \textbf{Польское королевство} и \textbf{Великое княжество Литовское}.

\hfill

Поводом для вмешательства в дела Речи Посполитой послужил вопрос о положении диссидентов (то есть некатолического меньшинства — православных и протестантов), чтобы те были уравнены с правами католиков. Екатерина оказывала сильное давление на шляхту с целью избрания на польский престол своего ставленника \textbf{Станислава Августа Понятовского}, который и был избран. Часть польской шляхты выступила против этих решений и организовала восстание, поднятое в Барской конфедерации. Оно было подавлено русскими войсками в союзе с польским королём. В 1772 году Пруссия и Австрия, опасаясь усиления российского влияния в Польше и её успехами в войне с Османской империей (Турция), предложили Екатерине провести \textbf{раздел Речи Посполитой} в обмен на прекращение войны, угрожая в противном случае войной против России. Россия, Австрия и Пруссия ввели свои войска.

\hfill

В 1772 году состоялся \textbf{Первый раздел Речи Посполитой}. Австрия получила всю Галицию с округами, Пруссия — Западную Пруссию (Поморье), Россия — \textbf{восточную часть Белоруссии до Минска (губернии Витебская и Могилёвская) и часть латвийских земель, входивших ранее в Ливонию}. Польский сейм был вынужден согласиться с разделом и отказаться от претензий на утраченные территории: Польшей было потеряно 380 000 км² с населением в 4 миллиона человек.

Польские дворяне и промышленники содействовали принятию Конституции 1791 года; консервативная часть населения Тарговицкой конфедерации обратилась к России за помощью.

\hfill

В 1793 году состоялся \textbf{Второй раздел Речи Посполитой}, утверждённый на Гродненском сейме. Пруссия получила Гданьск, Торунь, Познань (часть земель по р. Варта и Висла), Россия — \textbf{Центральную Белоруссию с Минском и Новороссии} (часть территории современной Украины).

В марте 1794 года началось \textbf{восстание под руководством Тадеуша Костюшко}, целями которого было восстановление территориальной целостности, суверенитета и Конституции 3 мая, однако весной того же года оно было \textbf{подавлено русской армией под командованием А. В. Суворова}. Во время восстания Костюшко восставшими поляками, захватившими русское посольство в Варшаве, были обнаружены документы, имевшие большой общественный резонанс, в соответствии с которыми король Станислав Понятовский и ряд членов Гродненского сейма в момент утверждения 2-го раздела Речи Посполитой получили деньги от русского правительства — в частности, Понятовский получил несколько тысяч дукатов.

\hfill

В 1795 году состоялся \textbf{Третий раздел Речи Посполитой}. Австрия получила Южную Польшу с Люблином и Краковом, Пруссия — Центральную Польшу с Варшавой, Россия — \textbf{Литву, Курляндию, Волынь и Западную Белоруссию}.

13 (24) октября 1795 года — \textbf{конференция трёх держав о падении польского государства}, оно потеряло государственность и суверенитет. 

\pagebreak
\subsection{Россия и революция во Франции (1789–1793): проблема взаимоотношений и взаимовлияния}

\subsubsection{Реакция в России на Великую Французскую Революцию}

Россия приняла \textbf{активное участие в войнах против революционной Франции}. Несмотря на неизбежные трудности и потери, связанные с европейскими походами русских войск, Россия как великая держава была на них обречена. В конкретно-исторических условиях конца $XVIII$ столетия эти войны были продиктованы стремлением императорской России предотвратить крушение существовавшего европейского порядка, защитить монархический принцип государственного устройства, который воспринимался как гарантия внутренней и международной стабильности. Однако Великая Французская революция 1789-1794 гг. не только привела к крушению абсолютизма во Франции, но и вместе с промышленной революцией в Англии означала вступление наиболее развитых стран Запада в индустриальную стадию развития.

\hfill

Начало революционных событий (1789 год) встревожили Екатерину II, которые напомнили ей события пугачевского бунта. Произошло \textbf{осложнение отношений тогда еще с королевской Францией}, после того, как Людовик XVI одобрил конституцию. Однако во время антимонархического выступления в Париже русский посол во Франции готовил побег короля из столицы, который не увенчался успехом. В январе 1793 г. в Париже был казнён Людовик XVI.

\hfill

Таким образом, в эти дни Французской революции в отношениях между странами наступил резкий перелом. \textbf{Россия стала одной из основательниц антифранцузской коалиции}, кроме того, были \textbf{разорваны торговые и любые дипломатические отношения}.

\subsubsection{Первая антифранцузская коалиция}

В её состав входили: \textbf{Великобритания, Пруссия, Неаполь, Тоскана, Австрия, Испания, Нидерланды, Российская империя}, произошла Французская революция. 14 июля восставшие с шумом овладели Бастилией. В стране установился буржуазный строй. В Петербурге начавшуюся революцию считали поначалу повседневным бунтом, вызванным временными финансовыми затруднениями и личными качествами короля Людовика XVI. С ростом революции в Петербурге начали опасаться распространения революции на все феодально-абсолютистские страны Европы. Опасения русского двора разделяли монархи Пруссии и Австрии.

\hfill

В 1790 году заключён \textbf{союз Австрии и Пруссии с целью военного вмешательства во внутренние дела Франции}, но ограничились разработкой планов интервенции и оказанием материальной помощи французской эмиграции и контрреволюционному дворянству внутри страны (Екатерина дала взаймы 2 млн руб. на создание наёмной армии).

\hfill

В марте 1793 года подписана \textbf{конвенция между Россией и Англией об обоюдном обязательстве оказывать друг другу помощь в борьбе против Франции}: закрывать свои порты для французских судов и препятствовать торговле Франции с нейтральными странами (Екатерина II отправила русские военные корабли в Англию для блокады французских берегов). 

\end{document}